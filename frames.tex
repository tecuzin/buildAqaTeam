%--------------------------------------
% Create title frame
\titleframe

%--------------------------------------
% Table of contents
\begin{frame}{Fil rouge}
  \setbeamertemplate{section in toc}[sections numbered]
  \tableofcontents%[hideallsubsections]
\end{frame}


%==============================================
\section{Introduction}
%==============================================
\begin{frame}{\insertsectionhead}
  \framesubtitle{Le contexte}

  \begin{itemize}
    \item Passer de testeur à lead QA afin de répondre l'augmentation du volume
    \item Accompagner la croissance de l'entreprise
    \item Développer l'activité qualité en interne et externe (prestations)
  \end{itemize}
\end{frame}

\begin{frame}{\insertsectionhead}
  \framesubtitle{Le profil de DATASOLUTION}
  \begin{itemize}
    \item ESN comptant environ 200 clients pour 300 employés
    \item Des agences réparties en France et dans le monde
    \item Une chaînec complète de production de solutions e-commerce
  \end{itemize}
\end{frame}

\begin{frame}{\insertsectionhead}
  \framesubtitle{La problématique}
  \begin{itemize}
    \item Augmenter le niveau de qualité des projets
    \item Fiabiliser les livraisons, donner confiance dans le fait de livrer plus vite
    \item Servir nos petits clients aussi bien que les plus importants
    \item Donner de la perspective au métier d'ingénieur qualité au sein de \datasolution
  \end{itemize}
\end{frame}

%==============================================
\section{Les aspects à prendre en compte}
%==============================================

\begin{frame}{\insertsectionhead}
  \framesubtitle{Le besoin !}
  \begin{itemize}
    \item Le besoin de qualité, c'est d'abord un besoin resenti par toute le équipes
    \item La mise en place de l'organisation répond à ce besoin
    \item La qualité du point de vue de l'équipe est donc vue comme un produit et un service
  \end{itemize}
\end{frame}

\begin{frame}{\insertsectionhead}
  \framesubtitle{La culture de l'entreprise}
  \begin{itemize}
    \item \datasolution est avant tout composée de développeurs, la qualité reprnd donc les codes propres à ce métier
    \item Le milieu des ESN, mais aussi les expériences très variées des collaborateurs sont à prendre en compte
    \item Les outils existants sont pour la plupart issus d'un développement interne et/ou d'une adaptation d'outils existants
  \end{itemize}
\end{frame}

\begin{frame}{\insertsectionhead}
  \framesubtitle{Les clients}
  \begin{itemize}
    \item Des clients très variés (B2B, B2C, de tailles très différentes
    \item Niveau d'Agilité et intégration de la QA très différents suivant les clients
    \item Budget très variés
  \end{itemize}
\end{frame}

\begin{frame}{\insertsectionhead}
  \framesubtitle{Les Hommes}
  \begin{itemize}
    \item la particularité des QA est d'être issus de formations et de domains très variés
    \item Pas de profil standard, mais une multitude de compétences complémentaires
    \item Des ambitions et une vision du métier qui varient fortement selon les individus
  \end{itemize}
\end{frame}

\begin{frame}{\insertsectionhead}
  \framesubtitle{Les outils}
  \begin{itemize}
    \item Du Word/Excel à la chaîne d'outils parafaitement intégrée
    \item Les contributeurs, équipes, et collaborateurs sont habitués à certaines stacks techniques et outils
    \item les outils peuvent demander de la mise en place, de la formation, et avoir un coût de mise en place et d'utilisation
  \end{itemize}
\end{frame}

\begin{frame}{\insertsectionhead}
  \framesubtitle{Le marché}
  \begin{itemize}
    \item Les cients et collaborateurs ont une idée intuitive de ce qu'ils attendent de la QA
    \item Cette vision correspond parfois à ce qui est visible dan les articles, chez des clients/concurrents ...
    \item 
  \end{itemize}
\end{frame}

\begin{frame}{\insertsectionhead}
  \framesubtitle{Le temps}
  \begin{itemize}
    \item Faire les choses "bien", cela prend du temps
    \item Répondre au besoin, rapidement n'st pas toujours compatibles avec ce temps
    \item Il est donc important de bien choisir les actions à réaliser et à quelle échance les réaliser pour que le niveau de qualité progresse au fil du temps, sans créer de rupture de production
  \end{itemize}
\end{frame}

\begin{frame}{\insertsectionhead}
  \framesubtitle{Les ambitions des collaborateurs}
  \begin{itemize}
    \item Chaque collaborateur apporte sa vision de la QA, et a une idée de ce vers quoi il souhaite évoluer
    \item Une équipe qui fonctionne, c'est une équipe dont les compétences sont complémentaires, mais également au sein de laquelle les personnes comprennent les difficiultés des autres membres et ont envie d'aider
    \item La communication, la formation et l'empathie pemettent de tendre vers des profits en T
  \end{itemize}
\end{frame}

\begin{frame}{\insertsectionhead}
  \framesubtitle{Les technologies utilisées par les équipes de développement}
  \begin{itemize}
    \item Au sein d'une entreprise, et chez les clients, mais aussi chez les nouveaux collaborateurs, les outils utilisés peuvent varier
    \item Une des tâches les plus complexes mais à forte valeur ajoutée consiste à trouver quel jeu d'outils utiliser pour produire de manière fiable, en limitant le temps d'apprentissage et de mise en place et en favorisant l'adaptabilité dans le temps
  \end{itemize}
\end{frame}

\begin{frame}{\insertsectionhead}
  \framesubtitle{Les technologies maîtrisées par les QA}
  \begin{itemize}
    \item Le niveau de maîtrise, mais aussi les sensibilités des membres de l'équipe varient beaucoup
    \item Les contextes variant beaucoup selon les projets, chaque personne verra son niveau de maîtrise évoluer de manière très différente
    \item Il est donc nécessaire de faire évoluer la qualité pour qu'elle puisse être portée par les QA
  \end{itemize}
\end{frame}

\begin{frame}{\insertsectionhead}
  \framesubtitle{Les possibilités de collaboration avec les autres acteurs}
  \begin{itemize}
    \item Chaque projet, équipe et client étant différent, la collaboration sera plus ou moins simple et possible aux différents stades es projets
    \item Les processus qualité et la composition de l'équipe doivent permettre aux QA d'apporter rapidement et au plus tôt un maximu de valeur ajoutée
  \end{itemize}
\end{frame}

\begin{frame}{\insertsectionhead}
  \framesubtitle{Les oportunités}
  \begin{itemize}
    \item Au vu des élément précédents, il est impossible de planifier à l'avance l'évolution d'une équipe de manière précise
    \item Cette équipe doit donc être capable de saisir des oportunités et de les utiliser pour faire progresser la qualité et la mettre au service des équipes et des clients
  \end{itemize}
\end{frame}

%==============================================
\section{La roadmap passée et avenir}
%==============================================

\begin{frame}{\insertsectionhead}
  \framesubtitle{L'étape 0 - Une volonté de test par les équipes}
    \begin{itemize}
      \item Constat 1 : Il ya toujours quelque part quelu'un qui teste
      \item Constat 2 : Il y a toujours une volonté de bien faire
      \item Constat 3 : La culture, les contraintes internes ou externes viennent souvent "frustrer" cette volonté de tester, satisfiare le lcient en bien faire
      \item L'étape 0 consiste donc à éviter le problème majeur du problème en production lorsque l'on ne teste pas. Même quand cela passe par de longues phases de test de régression
    \end{itemize}
\end{frame}

\begin{frame}{\insertsectionhead}
  \framesubtitle{Etape 1 - Donner un espace-temps au test}
     \begin{itemize}
      \item Au début, les tests sont très peu visibles, noyés ans les étapes de développement "quand on a le temps" et/ou réalisés en fin de cycle sans traçabilité
      \item Il faut donner aux tests un espace/temps c'est  dire les rendre visibles, mesurables, et identifiables pour pouvoir se rendre compte de ce qui est fait.
      \item Cet espace temps peut se matérialiser par des TNR, une colonne dans le tableau agile, une communication sur les TU, TI, TA mis en place pendant le développement. Un exemple très avancé et abouti est le TDD, qui rend visible dès le début les tests à réaliser
    \end{itemize}
\end{frame}

\begin{frame}{\insertsectionhead}
  \framesubtitle{Etape 2 - Identifier les axes techniques, humains et organisationnels}
 
    \begin{itemize}
      \item Une fois les tests visibles, les personnes impliquées, les moments, les éléments de l'organisation et les moyens deviennent visibles, avec leurs painpoints et points fort
      \item On a donc une passe pour réfléchir à comment définir développer rapidement une "offre" interne de tests à destination des équipes, clients et de l'équipe QA
      \item Cette étape cruciale demande du temps et beaucoup de collaboration
       \item Une fois cette étape réussie, il est possible de la "vendre"
    \end{itemize}
\end{frame}

\begin{frame}{\insertsectionhead}
  \framesubtitle{Etape 3 - Développer une offre et convaincre}
    \begin{itemize}
      \item Une offre, c'est finalement une valeur ajoutée que l'on souhaite apporter à un ou un groupe de "clients" en leur demandant  en échange un effort
      \item Le porteur de l'offre souhaite que son offre soit valorisée au maximum, le client lui souhaite toujours minimiser son effort
      \item L'offre, et donc la stratégie de développement de l'équipe QA doit être mise en place et présentée de manière à apporter de la valeur ajoutée pour chaque profil de destinataire. Il faut pour cela comprendre le besoin (plus de sérnité, meilleure relation client, maintenabilité accrue, ...)
      \item A cette étape il est aussi important d'identifier les "domaines" à développer
      \item Attention à bien mettre en face de cette offre les moyens qui permettront de la mettre en oeuvre et de la faire évoluer. Vendre des tests autos sans automaticien est risqué ;)
    \end{itemize}
\end{frame}

\begin{frame}{\insertsectionhead}
  \framesubtitle{Etape 4 - Former, recruter, développer}
    \begin{itemize}
      \item Une fois l'offre développée, il faut faire correspondre l'équipe à l'offre, mais aussi visualiser comment l'offre pourra évoluer en fonction de l'équipe
      \item Le recrutement et la formation des équipes permettent donc à la fois de répondre aux demandes des clients, mais aussi d'ouvrir des perspectives pour développer l'offre dans le temps
      \item On cherchera également à développer la capacité des membres de l'équipe à collaborer et comprendre l'autre, afin de consolider la cohésion et la bonnen diffusion de la culture qualité et de ce qu'elle implique
    \end{itemize}
\end{frame}

\begin{frame}{\insertsectionhead}
  \framesubtitle{Etape 5 - Structurer l'équipe}
  
    \begin{itemize}
      \item Une fois les différents domaines identifiés, l'équipe peut se strucutrer en fonction de groupes de projets ou de domaines, et choisir ses "leads" pour chacun d'entre eux, l'idéal étant de les laisser émerger
      \item L'équipe se structurera ensuite naturellement autour de ces leads en fonction de ces affinités, l'important étant de bien créer les oportunités d'échanges
    \end{itemize}
\end{frame}

\begin{frame}{\insertsectionhead}
  \framesubtitle{Etape 6 - Etendre le champ d'action}
 
    \begin{itemize}
      \item C'est à dire itérer les étapes précédentes ;)
    \end{itemize}
\end{frame}

\begin{frame}{\insertsectionhead}
  \framesubtitle{Etape 7 - Impliquer les équipes de développement}
    \begin{itemize}
     \item Lors du développement, impliquer tous les acteurs du développement
     \item Récolter les feedbacks, analyser les échecs ou les difficultés
     \item Communiquer communiquer communiquer
    \end{itemize}
\end{frame}

%==============================================
\section{Le profil de l'équipe}
%==============================================

\begin{frame}{\insertsectionhead}
  \framesubtitle{Tshaped}
    \begin{itemize}
      \item Chercher la compétence motivante et maîtrisée chez chacun
      \item chercher les compétences et apétences
          \end{itemize}
\end{frame}

\begin{frame}{\insertsectionhead}
  \framesubtitle{Ouverte à la collaboration}
 
    \begin{itemize}
      \item Aller vers les autres, toujours
    \end{itemize}
\end{frame}

\begin{frame}{\insertsectionhead}
  \framesubtitle{Orientée client}
    \begin{itemize}
      \item Former les collaborateurs à l'approche client, et sensibiliser le client à l'approche inclusive
    \end{itemize}
\end{frame}

\begin{frame}{\insertsectionhead}
  \framesubtitle{Vision long terme}
    \begin{itemize}
      \item Trouver ce qui fera progresser sur le long terme
      \item Sensibiliser sur cette vision, l'expliquer
      \item Construire la roadmap avec les autres acteurs pour atteindre cet horizon
      \item Ajuster l'horizon à chaque pas
    \end{itemize}
\end{frame}

\begin{frame}{\insertsectionhead}
  \framesubtitle{Evolutive}

    \begin{itemize}
      \item Former, communiquer
      \item Ecouter, capter chaque occasion de faire progresser l'équipe
    \end{itemize}
\end{frame}

\begin{frame}{\insertsectionhead}
  \framesubtitle{Evolutive}
    \begin{itemize}
    \item Esayer d'avancer en permanence, même par petits pas
    \item Choisir et aider l'équipe dans ce sens
    \end{itemize}
\end{frame}

%==============================================
\section{La stack technique}
%==============================================

\begin{frame}{\insertsectionhead}
  \framesubtitle{OpenSource}
    \begin{itemize}
      \item Correspond bien à la culture de l'entreprise et au besoin de faire évoluer les outils en fonction de nos besoins.
      \item Possibilité d'ajouter des utilisateurs sans limitation, utile pour inclure les clients dans les projets
    \end{itemize}
\end{frame}

\begin{frame}{\insertsectionhead}
  \framesubtitle{Evolutive}
    \begin{itemize}
      \item Maintenable, facile d'accès pour les QA et les autres acteurs
      \item Communiquer sur ls choix
    \end{itemize}
\end{frame}

\begin{frame}{\insertsectionhead}
  \framesubtitle{Accessible}
    \begin{itemize}
      \item langage de programmation : Groovy, JS, Python : inclusifs vis à vis des développeurs et des QA non automaticiens
      \item Installation aisée : exemple de Cypress et des outils bassés sur une stack node.js
    \end{itemize}
\end{frame}

\begin{frame}{\insertsectionhead}
  \framesubtitle{Robustes}
    \begin{itemize}
      \item langage de programmation : Groovy, JS, Python : inclusifs vis à vis des développeurs et des QA non automaticiens
      \item Installation aisée : exemple de Cypress et des outils bassés sur une stack node.js
    \end{itemize}
\end{frame}

\begin{frame}{\insertsectionhead}
  \framesubtitle{Maintenus}
    \begin{itemize}
      \item langage de programmation : Groovy, JS, Python : inclusifs vis à vis des développeurs et des QA non automaticiens
      \item Installation aisée : exemple de Cypress et des outils bassés sur une stack node.js
    \end{itemize}
\end{frame}

\begin{frame}{\insertsectionhead}
  \framesubtitle{Maintenables}
    \begin{itemize}
      \item Produire un code maintenable et permettant de profiter des oportunités (développeurs ayant du temps pour les projets internes, stagiaires, ...
      \item Produire de la documentation associée à la stack, mais aussi former et communiquer sur les solution mises en place afin de diffuser les connaissances
    \end{itemize}
\end{frame}

\begin{frame}{\insertsectionhead}
  \framesubtitle{Interconnectables}
    \begin{itemize}
      \item langage de programmation : Groovy, JS, Python : inclusifs vis à vis des développeurs et des QA non automaticiens
      \item Installation aisée : exemple de Cypress et des outils bassés sur une stack node.js
    \end{itemize}
\end{frame}

%==============================================
\section{Conclusions et perspectives}
%==============================================
\begin{frame}{Conclusion}

    \begin{itemize}
      \item Un "long chemin" en quatre ans, où chaque étape demande de l'agilité, de la collaboration et la prise en compte des besoins des équipes et des clients
    \end{itemize}

\end{frame}

\section{Merci de votre attention\\Des questions ?}
